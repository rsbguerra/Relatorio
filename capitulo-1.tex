\chapter{Introdução}
\label{ch::intro}


\section{Enquadramento}
\label{sec::intro:enquad}


A história da captura, armazenamento e visualização de imagens é extremamente rica e milenar. Marcos importantes destacam-se, sendo do particular interesse no Século XXI os grandes passos dados na imagem digital.

Contudo, a vasta maioria da fotografia tem-se centrado na captura de imagens estáticas em duas dimensões. O interesse na captura e representação de objetos e momentos em três dimensões tem ganho um interesse crescente nas últimas décadas.

A área dedicada ao estudo deste modelo, a \textbf{holografia}, carece de vários marcos que já fazem parte do quotidiano da fotografia clássica, nomeadamente padrões \textit{standardizados} para a codificação e compressão de \textbf{hologramas} em formato digital.


\section{Motivação}
\label{sec::intro:motiv}
Dada a referida ausência de \textit{standards} no armazenamento e representação da informação, reconstrução e codificação de um holograma, é do interesse da comunidade do JPEG Pleno estudar os codificadores existentes para melhor perceber qual a sua adaptabilidade aos hologramas e quais as modificações necessárias para resolver a falta de padrões nos pontos mencionados.


\section{Objetivos}
\label{sec::intro:objs}
Tendo em mente a motivação apresentada na secção \ref{sec::intro:motiv}, o presente projeto tem por objetivo principal investigar o desempenho do codec JPEG2000 na codificação de hologramas a cores em multivistas.

Por seu turno, os objetivos secundários --- os quais refletem as diferentes fases da investigação --- são os seguintes:

\begin{enumerate}
  \item \label{obj:implementar_reconstrutor} Implementar um reconstrutor para hologramas com cor;
  \item \label{obj:reconstruir_vistas} Reconstruir hologramas em vários pontos de vista;
  \item \label{obj:comprimir} Comprimir hologramas reconstruidos com recurso ao codificador JPEG2000;
  \item \label{obj:avaliar_psnr} Avaliar a qualidade das imagens comprimidas face à reconstrução original.
\end{enumerate}

Os objetivos supra-mencionados refletem o objetivo geral de estudar holografia e, assim, expandir o conhecimento na área das tecnologias multimédia.


\section{Organização do Documento}
\label{sec::intro:organiza}

Apresenta-se de seguida a estrutura do presente documento:

\begin{itemize}
  \item Primeiramente, no \ref{ch::intro}\textordmasculine~capítulo --- \textbf{Introdução} ---, são formulados os objetivos do presente projeto, assim como a motivação deste;

  \item No \ref{ch::estado-arte}\textordmasculine~capítulo --- \textbf{Estado da Arte} --- TODO;

  \item No \ref{ch::tecno-ferr}\textordmasculine~capítulo --- \textbf{Tecnologias e Ferramentas Utilizadas} --- são expostas as tecnologias utilizadas para a investigação levada a cabo no âmbito no projeto (\textit{software} externo e linguagem utilizada para a implementação de \textit{scripts} próprios), assim como os materiais a que se recorreu (em particular os hologramas testados e as especificações dos computadores utilizados);

  \item No \ref{ch::imp-test}\textordmasculine~capítulo --- \textbf{Etapas de Desenvolvimento e Implementação} --- é dada a conhecer a estratégia de investigação, detalhando as três partes na qual se dividiu e enumerando os respetivos objetivos;

  \item No \ref{ch::test-result}\textordmasculine~capítulo --- \textbf{Testes e Resultados} --- procede-se à apresentação dos resultados obtidos, sendo feita uma discussão destes com base em três questões que se levantaram após a sua análise;

  \item Por fim, no \ref{ch::conclusao}\textordmasculine~capítulo --- \textbf{Conclusões e Trabalho Futuro} ---, são formuladas as conclusões do projeto, rematando assim os objetivos apresentados na Secção \ref{sec::intro:objs}, e são feitas propostas de estudos para trabalhos futuros.
\end{itemize}

