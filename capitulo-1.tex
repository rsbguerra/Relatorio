\chapter{Introdução}
\label{chap:intro}

\section{Enquadramento}
\label{sec:amb}

A história da captura, armazenamento e visualização de imagens é extremamente rica e milenar. Marcos importantes destacam-se, sendo do particular interesse no Século XXI os grandes passos dados na imagem digital.

Contudo, a vasta maioria da fotografia tem-se centrado na captura de imagens estáticas em duas dimensões. O interesse na captura e representação de objetos e momentos em três dimensões tem ganho um interesse crescente nas últimas décadas.

A área dedicada ao estudo deste modelo, a \textbf{holografia}, carece de vários marcos que já fazem parte do quotidiano da fotografia clássica, nomeadamente padrões \textit{standardizados} para a codificação e compressão de \textbf{hologramas} em formato digital.


\section{Motivação}
\label{sec:mot}
Dada a referida ausência de \textit{standards} no armazenamento e representação da informação, reconstrução e codificação de um holograma, é do interesse da comunidade do JPEG Pleno estudar os codificadores existentes para melhor perceber qual a sua adaptabilidade aos hologramas e quais as modificações necessárias para resolver a falta de padrões nos pontos mencionados.


\section{Objetivos}
\label{sec:obj}
%Neste projeto pretende-se estudar imagens holográficas e o codificador JPEG2000 na codificação de imagens holográficas. Investigar o comportamento do codificador JPEG2000 na codificação da cor quando a codificação é efetuada em diferentes vistas.

Tendo em mente a motivação apresentada na secção \ref{sec:mot}, o presente projeto tem por objetivo principal investigar o desempenho do codec JPEG2000 na codificação de hologramas a cores em multivistas.

Por seu turno, os objetivos secundários --- os quais refletem as diferentes fases da investigação --- são os seguintes:

\begin{enumerate}
  \item Implementar um reconstrutor para hologramas com cor;
  \item Reconstruir hologramas em vários pontos de vista;
  \item Comprimir hologramas reconstruidos com recurso ao codificador JPEG2000;
  \item Avaliar a qualidade das imagens comprimidas face à reconstrução original.
\end{enumerate}

Os objetivos supra-mencionados refletem o objetivo geral de estudar holografia e, assim, expandir o conhecimento na área das tecnologias multimédia.


\section{Organização do Documento}
\label{sec:organ}
