\chapter{Introdução}
\label{chap:intro}

\section{Enquadramento}
\label{sec:amb} 

\begin{comment}
  Os acrónimos devem ser definidos recorrendo ao pacote (\emph{package}) \texttt  {acronym}, usando os comandos \texttt{\textbackslash acro}, \texttt {\textbackslash ac}, \texttt{\textbackslash acp}, etc. E.g., \emph{The   subject of this report is network protocols, namely \ac{TCP}.  \ac{TCP} is  studied for several aspects of performance.}

  Este relatório foi feito no contexto da unidade curricular de projeto da \ac  {UBI}. Foi na \ac{UBI} que desenvolvi todo o trabalho. \ac{CFIUTE}
\end{comment}

A história da captura, armazenamento e visualização de imagens é extremamente rica e milenar. Marcos importantes destacam-se, sendo do particular interesse no Século XXI os grandes passos dados na imagem digital.

Contudo, a vasta maioria da fotografia tem-se centrado na captura de imagens estáticas em duas dimensões. O interesse na captura e representação de objetos e momentos em três dimensões tem ganho um interesse crescente nas últimas décadas.

A área dedicada ao estudo deste modelo, a \textbf{holografia}, carece de vários marcos que já fazem parte do quotidiano da fotografia clássica, nomeadamente padrões \textit{standardizados} para a codificação e compressão de \textbf{hologramas} em formato digital.


\section{Motivação}
\label{sec:mot}
Dada a referida ausência de \textit{standards} no armazenamento e representação da informação, reconstrução e codificação de um holograma, é do interesse da comunidade do JPEG Pleno estudar os codificadores existentes para melhor perceber qual a sua adaptabilidade aos hologramas e quais as modificações necessárias para resolver a falta de padrões nos pontos mencionados.

% ja estou inscrita a mestrado pls let me in


\section{Objetivos}
\label{sec:obj}
%Neste projeto pretende-se estudar imagens holográficas e o codificador JPEG2000 na codificação de imagens holográficas. Investigar o comportamento do codificador JPEG2000 na codificação da cor quando a codificação é efetuada em diferentes vistas.

Tendo em mente a motivação apresentada na secção \ref{sec:mot}, o presente projeto tem por objetivo principal investigar o desempenho do codec JPEG2000 na codificação de hologramas a cores em multivistas.

Por seu turno, os objetivos secundários --- os quais refletem as diferentes fases da investigação --- são os seguintes:

\begin{enumerate}
  \item Implementar um reconstrutor para hologramas com cor;
  \item Comprimir hologramas reconstruidos com recurso ao codificador JPEG2000;
  \item Avaliar a qualidade da imagem comprimida face ao holograma original.
\end{enumerate}

Os objetivos supra-mencionados refletem o objetivo geral de estudar holografia e, assim, expandir o conhecimento na área das tecnologias multimédia.


\section{Organização do Documento}
\label{sec:organ}

\begin{comment}
% !POR EXEMPLO!
De modo a refletir o trabalho que foi feito, este documento encontra-se estruturado da seguinte forma:
\begin{enumerate}
\item O primeiro capítulo -- \textbf{Introdução} -- apresenta o projeto, a motivação para a sua escolha, o enquadramento para o mesmo, os seus objetivos e a respetiva organização do documento.
\item O segundo capítulo -- \textbf{Tecnologias Utilizadas} -- descreve os conceitos mais importantes no âmbito deste projeto, bem como as tecnologias utilizadas durante do desenvolvimento da aplicação.
\item ...
\end{enumerate}
\end{comment}

\begin{comment}
\section{Algumas Dicas -- [RETIRAR DA VERSÃO FINAL]}
ALGUMAS DICAS
Os relatórios de projeto são individuais e preparados em \LaTeX, seguindo o formato disponível na página da unidade curricular. Deve ser prestada especial atenção aos seguintes pontos:
\begin{enumerate}
  \item O relatório deve ter um capítulo Introdução e Conclusões e Trabalho Futuro (ou só Conclusões);
  \item A última secção do primeiro capítulo deve descrever sucintamente a organização do documento;
  \item O relatório pode ser escrito em Língua Portuguesa ou Inglesa;
  \item Todas as imagens ou tabelas devem ter legendas e ser referidas no texto (usando comando \texttt{\textbackslash ref\{\}}).
\end{enumerate}
\end{comment}