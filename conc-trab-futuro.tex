\chapter{Conclusões e Trabalho Futuro}
\label{ch::conclusao}

\section{Conclusões Principais}
\label{sec::conclusao:principal}

%Esta secção contém a resposta à questão: \\
%\emph{Quais foram as conclusões princípais a que o(a) aluno(a) chegou no fim deste trabalho?}

TODO

Após os resultados expostos na Secção \ref{sec::test-result:result} e a sua respetiva discussão na Secção \ref{sec::test-result:discussao}, podem-se enumerar \textbf{X} conclusões:

\begin{enumerate}
    \item O recurso à transformada de cor não é desejável;
    \item O uso de maiores \textit{bitrates} produz hologramas comprimidos de melhor qualidade;
    \item O uso de maiores \textit{bitrates} produz hologramas comprimidos de qualidade mais consistentes para diferentes vistas.
\end{enumerate}

% Para os hologramas estudados, propõe-se o uso de um \textit{bitrate} mínimo de \SI{2.5}{} para a compressão com recurso ao formato JPEG2000.


% QUESTÕES:
% Há algum débito mínimo comum aceitável?
% Os hologramas comprimidos são de qualidade aceitável?
% ...




\section{Trabalho Futuro}
\label{sec::conclusao:futuro}

%Esta secção responde a questões como:\\
%\emph{O que é que ficou por fazer, e porque?}\\
%\emph{O que é que seria interessante fazer, mas não foi feito por não ser exatamente o objetivo deste trabalho?}\\
%\emph{Em que outros casos ou situações ou cenários -- que não foram estudados no contexto deste projeto por não ser seu objetivo -- é que o trabalho aqui descrito pode ter aplicações interessantes e porque?}

TODO

A par do concluído na Secção \ref{sec::conclusao:principal}, propõe-se o estudo alargado a mais hologramas para se poder determinar se existe um \textit{bitrate} mínimo comum  aceitável ou se este deve sempre depender do holograma original.

No mesmo sentido, caso o débito dependa do holograma, é plausível a pertinência de um estudo sobre métodos e métricas para determinação ótima do débito aquando da compressão.

