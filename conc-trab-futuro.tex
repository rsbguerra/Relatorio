\chapter{Conclusões e Trabalho Futuro}
\label{ch::conclusao}

\section{Conclusões Principais}
\label{sec::conclusao:principal}

Com o término da investigação apresentada, foram alcançadas as seguintes etapas:
\begin{itemize}
    \item Implementação de um conjunto de \textit{scripts} para a reconstrução, compressão e descompressão de hologramas com cor em multivista;
    \item Efetiva reconstrução de 3 hologramas em 16 vistas distintas;
    \item Utilização do codificador JPEG2000 para a compressão dos hologramas reconstruídos;
    \item Estudo dos hologramas comprimidos com a métrica \ac{PSNR}.
\end{itemize}

A frutífera conclusão destas etapas permitiu o alcance dos objetivos inicialmente traçados para este projeto.

Neste sentido, e após os resultados expostos na Secção \ref{sec::test-result:result} e a sua respetiva discussão na Secção \ref{sec::test-result:discussao}, podem-se enumerar as seguintes conclusões acerca do uso do formato JPEG2000 para a compressão de hologramas no plano do objeto:

\begin{enumerate}
    \item O recurso à transformada de cor não é desejável;
    \item O uso de maiores \textit{bitrates} produz hologramas comprimidos de melhor e mais consistente qualidade para diferentes vistas.
    \item Considerando que um \ac{PSNR} entre \SI{30}{\decibel} e \SI{50}{\decibel} é assumido como um intervalo que indica qualidade aceitável (\textit{i.e.} perdas visualmente impercetíveis) para multimédia comprimida em formatos com perda\cite{welstead1999fractal,barni2006document}, débitos tão baixos quanto \SI{0.3}{} revelaram-se suficientes para uma compressão satisfatória.
\end{enumerate}

Por consequência, e a fim de responder ao objetivo primário do presente projeto, conclui-se que o formato JPEG2000, no âmbito do Projeto JPEG Pleno, é adequado para a compressão de hologramas com cor em multivista.


\section{Trabalho Futuro}
\label{sec::conclusao:futuro}

No âmbito do projeto inicialmente proposto, foram cumpridos todos os objetivos delineados.

Contudo, durante a investigação, foi notória a ineficiência do uso de uma única \textit{thread} conforme implementado nos \textit{scripts} desenvolvidos. Seria, portanto, interessante o uso de \textit{multithreading} a fim de alcançar um menor tempo de execução e uma vasta melhoria no uso dos recursos disponibilizados pelos \textit{hardwares} utilizados (mencionados na Secção \ref{ssec::tecno-ferr:materiais:hardware}). Outra alternativa seria o uso de computação paralela com recurso a placas gráficas, a par da técnica utilizada pelos investigadores do \textit{Institute of Research \& Technology b<>com} (\textit{e.g.} com recurso à tecnologia \ac{CUDA}\cite{Gilles2016}).

A par do concluído na Secção \ref{sec::conclusao:principal}, propõe-se o estudo alargado a mais hologramas para se poder determinar se existe um \textit{bitrate} mínimo comum  aceitável ou se este deve sempre depender do holograma original.

No mesmo sentido, caso o débito dependa do holograma, é plausível a pertinência de um estudo sobre métodos para determinação ótima do débito aquando da compressão.

Ademais, não tendo sido objetivo do presente projeto, é de alto interesse o estudo da redução do \textit{bitrate} utilizado na compressão a fim de alcançar hologramas comprimidos sem perda de qualidade percetível e um tamanho de ficheiro o mais reduzido possível.

Por fim, o estudo da determinação das vistas com \ac{PSNR} ótimo é uma natural extensão do presente projeto. Tendo em conta que o uso de multivista não levou a uma melhoria na utilização da transformada de cor, a redução do \textit{speckle} (ruído) pode ser estudada a fim de verificar se existe melhoria dos resultados com o seu uso.
