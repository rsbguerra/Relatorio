\chapter{Etapas de Desenvolvimento e Implementação}
\label{chap:imp-test}

\section{Introdução}
\label{chap4:sec:intro}

\section{Reconstrução dos Hologramas}
\label{chap4:sec:...}

O projeto iniciou-se com uma pesquisa exaustiva sobre a ciência da holografia, a qual resultou no Estado da Arte resumido no Capítulo \ref{chap:estado-da-arte}.

Simultaneamente, foi efetuado um estudo das funções do \textit{software} desenvolvido no âmbito do projeto JPEG Pleno a fim de se poder fazer a respetiva transcrição para Python.

\begin{table}[!htbp]
    \centering
    \caption{Resumo da documentação da função \texttt{load\_hologram}.}
    \label{tab:load_hologram}
    \begin{tabular}{p{1cm} p{10cm}}
        \hline
        \multicolumn{2}{l}{\bfseries Nome da função}\\
         & \verb|load_hologram|\\
        \hline
        \multicolumn{2}{l}{\bfseries Protótipo original em MATLAB}\\
         & \mintinline[breaklines]{matlab}{function [hologram] = load_hologram(ampli_path, phase_path)}\\
        \hline
        \multicolumn{2}{l}{\bfseries Protótipo transcrito em Python}\\
         & \mintinline{python}{def load_hologram(ampli_path, phase_path)} \\
        \hline\multicolumn{2}{l}{\bfseries Descrição}\\
         & Esta função carrega um holograma da base de dados b<>com a partir dos seus ficheiros de amplitude e fase.\\
        \hline\multicolumn{2}{l}{\bfseries \textit{Inputs}}\\
         & \verb|ampli_path|: Diretório do ficheiro da imagem da amplitude (caminho relativo ou absoluto).\\
         & \verb|phase_path|: Diretório do ficheiro da imagem da fase (caminho relativo ou absoluto).\\
        \hline\multicolumn{2}{l}{\bfseries \textit{Output}}\\
         & Modulação complexa do holograma (3 canais: R-G-B).\\
        \hline
    \end{tabular}
\end{table}


\begin{table}[!htbp]
    \centering
    \caption{Resumo da documentação da função \texttt{propagate\_asm}.}
    \label{tab:propagate_asm}
    \begin{tabular}{p{1cm} p{10cm}}
        \hline
        \multicolumn{2}{l}{\bfseries Nome da função}\\
         & \verb|propagate_asm|\\
        \hline
        \multicolumn{2}{l}{\bfseries Protótipo original em MATLAB}\\
         & \mintinline[breaklines]{matlab}{function [v] = propagate_asm(u, pitch, wavelength, z)}\\
        \hline
        \multicolumn{2}{l}{\bfseries Protótipo transcrito em Python}\\
         & \mintinline[breaklines]{python}{def propagate_asm(u, pitch, wavelength, z)} \\
        \hline\multicolumn{2}{l}{\bfseries Descrição}\\
         & Esta função simula a propagação no plano complexo \verb|u| sobre a distância \verb|z| utilizando o \ac{ASM}.\\
        \hline\multicolumn{2}{l}{\bfseries \textit{Inputs}}\\
         & \verb|u|: Campo de onda de luz do plano de \textit{input} (um canal).\\
         & \verb|pitch|: Distância entre pixeis (em metros).\\
         & \verb|wavelength|: Comprimento de onda do canal de cor a propagar (em metros).\\
         & \verb|z|: Distância de propagação ao longo do eixo ótico (em metros).\\
        \hline\multicolumn{2}{l}{\bfseries \textit{Output}}\\
         & Campo de onda de luz no plano de destino (um canal).\\
        \hline
    \end{tabular}
\end{table}


\begin{table}[!htbp]
    \centering
    \caption{Resumo da documentação da função \texttt{reconst\_hologram}.}
    \label{tab:reconst_hologram}
    \begin{tabular}{p{1cm} p{10cm}}
        \hline
        \multicolumn{2}{l}{\bfseries Nome da função}\\
         & \verb|reconst_hologram|\\
        \hline
        \multicolumn{2}{l}{\bfseries Protótipo original em MATLAB}\\
         & \mintinline[breaklines]{matlab}{function [recons] = reconsHologram(hologram, pitch, wavelengths, z, pupilPos, pupilSize)
         }\\
        \hline
        \multicolumn{2}{l}{\bfseries Protótipo transcrito em Python}\\
         & \mintinline[breaklines]{python}{def reconst_hologram(hologram, pitch, wavelengths, z, pupil_pos=[0,0], pupil_size=None)} \\
        \hline\multicolumn{2}{l}{\bfseries Descrição}\\
         & Esta função reconstrói o holograma a uma distância \verb|z|, utilizando o \ac{ASM}. Permite o uso de uma janela para obter reconstruções de diferentes pontos de vista.\\
        \hline\multicolumn{2}{l}{\bfseries \textit{Inputs}}\\
         & \verb|hologram|: Holograma de modulação complexa (3 canais: R-G-B). \\
         & \verb|pitch|: Distância entre pixeis (em metros).\\
         & \verb|wavelengths|: Comprimentos de onda de luz (em metros, 3 canais: R-G-B).\\
         & \verb|z|: Distância de reconstrução (em metros).\\
         & \verb|pupilPos|: Posição da janela (em pixeis, canto superior direito).\\
         & \verb|pupilSize|: Tamanho da janela (em pixeis, altura $\times$ largura).\\
        \hline\multicolumn{2}{l}{\bfseries \textit{Output}}\\
         & Reconstrução numérica do holograma (3 canais: R-G-B).\\
        \hline
    \end{tabular}
\end{table}

% - Estudo holografia
% - estudo funções matlab
% transcrever funções para scripts em python
% comparar output entre funções de matlab e python para confirmar que os hologramas estão a ser reconstruidos corretamente
% desenvolver script para reconstruir o mesmo holograma de 16 vistas diferentes


\section{Compressão dos Hologramas Reconstruídos}
\lipsum[1]
% automatização de execução dos softwares
% estudar o uso de kdu
% implementar script que comprime com e sem transformada de cor, e com bitrates: [0.1, 0.3, 0.6, 1.0, 1.5, 2.0, 2.5, 3.0, 3.5, 4.0, 4.5, 5.0], e descomprime as reconstruções do holograma

\section{Determinação das Métricas de Compressão}
\lipsum[1]
% calcular o débito com a métrica PSNR entre o holograma original e a imagem comprimida
% estudar qual a melhor forma de guardar os débitos calculados
% implementar no script a escrita dos dados num ficheiro json
% a partir dos ficheiros json gerados, criar gráficos para apresentar os resultados com uso da biblioteca matplotlib.pyplot








\section{Conclusões}
\label{chap4:sec:concs}
